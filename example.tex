% Options for packages loaded elsewhere
\PassOptionsToPackage{unicode}{hyperref}
\PassOptionsToPackage{hyphens}{url}
%
\documentclass[jou]{apa7}
\usepackage{amsmath,amssymb}
\usepackage{lmodern}
\usepackage{iftex}
\ifPDFTeX
  \usepackage[T1]{fontenc}
  \usepackage[utf8]{inputenc}
  \usepackage{textcomp} % provide euro and other symbols
\else % if luatex or xetex
  \usepackage{unicode-math}
  \defaultfontfeatures{Scale=MatchLowercase}
  \defaultfontfeatures[\rmfamily]{Ligatures=TeX,Scale=1}
\fi
% Use upquote if available, for straight quotes in verbatim environments
\IfFileExists{upquote.sty}{\usepackage{upquote}}{}
\IfFileExists{microtype.sty}{% use microtype if available
  \usepackage[]{microtype}
  \UseMicrotypeSet[protrusion]{basicmath} % disable protrusion for tt fonts
}{}
\makeatletter
\@ifundefined{KOMAClassName}{% if non-KOMA class
  \IfFileExists{parskip.sty}{%
    \usepackage{parskip}
  }{% else
    \setlength{\parindent}{0pt}
    \setlength{\parskip}{6pt plus 2pt minus 1pt}}
}{% if KOMA class
  \KOMAoptions{parskip=half}}
\makeatother
\usepackage{xcolor}
\IfFileExists{xurl.sty}{\usepackage{xurl}}{} % add URL line breaks if available
\IfFileExists{bookmark.sty}{\usepackage{bookmark}}{\usepackage{hyperref}}
\hypersetup{
  pdftitle={Example document with some references},
  pdfkeywords={LaTeX, RMarkdown, Document formatting},
  hidelinks,
  pdfcreator={LaTeX via pandoc}}
\urlstyle{same} % disable monospaced font for URLs
\setlength{\emergencystretch}{3em} % prevent overfull lines
\providecommand{\tightlist}{%
  \setlength{\itemsep}{0pt}\setlength{\parskip}{0pt}}
\setcounter{secnumdepth}{-\maxdimen} % remove section numbering
\ifLuaTeX
  \usepackage{selnolig}  % disable illegal ligatures
\fi
\newlength{\cslhangindent}
\setlength{\cslhangindent}{1.5em}
\newlength{\csllabelwidth}
\setlength{\csllabelwidth}{3em}
\newenvironment{CSLReferences}[2] % #1 hanging-ident, #2 entry spacing
 {% don't indent paragraphs
  \setlength{\parindent}{0pt}
  % turn on hanging indent if param 1 is 1
  \ifodd #1 \everypar{\setlength{\hangindent}{\cslhangindent}}\ignorespaces\fi
  % set entry spacing
  \ifnum #2 > 0
  \setlength{\parskip}{#2\baselineskip}
  \fi
 }%
 {}
\usepackage{calc}
\newcommand{\CSLBlock}[1]{#1\hfill\break}
\newcommand{\CSLLeftMargin}[1]{\parbox[t]{\csllabelwidth}{#1}}
\newcommand{\CSLRightInline}[1]{\parbox[t]{\linewidth - \csllabelwidth}{#1}\break}
\newcommand{\CSLIndent}[1]{\hspace{\cslhangindent}#1}

\title{Example document with some references}
\shorttitle{Example R Markdown APA doc}
\leftheader{Dainer-Best et al.}
\authorsnames[{1},{2},{2,3}]{Justin Dainer-Best,Second Author,Third
Author}
\authorsaffiliations{{Bard College}
,{Other Institution}
,{Third Institution}
}
\authornote{This is an author's note; you could remove it if not
desired. You could include a corresponding author, and their email.}
\date{March 29, 2021}
\abstract{This is an example of creating an approximately APA-style
manuscript using R Markdown and .bib files}
\keywords{LaTeX, RMarkdown, Document formatting}

\begin{document}
\maketitle

\hypertarget{introduction}{%
\section{Introduction}\label{introduction}}

This is a simple example of using R Markdown documents to create
APA-formatted documents with LaTeX compiling them into PDFs. You will
need to install a version of LaTeX to compile; some suggestions are
included
\href{https://bookdown.org/yihui/rmarkdown-cookbook/install-latex.html}{here},
but if they do not work, you may want to install
\href{http://tug.org/mactex/}{MacTeX} for Macs or
\href{https://miktex.org/}{MiKTeX} for PCs. Of course, the Rmd document
can also be compiled (knit) as an HTML file if you do not want to
install a TeX distribution, but it will not follow APA style.

You will have to compile your references in a bibliography. Most folks
recommend using Zotero, as do I; you'll need to output your references
to a .bib file. Document/article titles will follow the capitalization
you've got---use APA style yourself! (The most recent versions of R
Markdown can also
\href{https://rmarkdown.rstudio.com/authoring_bibliographies_and_citations.html}{help
add your references}, including just from a DOI link.)

Some packages (e.g., \{\href{https://github.com/crsh/papaja}{papaja}\};
(\protect\hyperlink{ref-austbarth2020}{Aust \& Barth, 2020})) will
create APA-draft manuscripts, but may introduce extra details beyond the
basic style. The style I lay out here will simply create a document
using APA formatting of references.

\hypertarget{installation}{%
\section{Installation}\label{installation}}

You will need to download the \texttt{apa.csl} and \texttt{template.tex}
documents into a directory where your Rmd file lives. Open (or create)
your R Markdown document.

Your document will automatically include some YAML headers:

\begin{verbatim}
---
title: "Your title"
author: "Your name"
date: "3/29/2021"
output: pdf_document
---
\end{verbatim}

You can keep the title and date, but you will need to update the others
headers to include some of the following YAML headers in your R Markdown
document. In particular, you must include the \texttt{doctype}, the full
set of \texttt{output} parameters
(\texttt{output:\ pdf\_document:\ template:\ "template.tex"}), and the
\texttt{bibliography} and \texttt{csl} lines. The others, including
\texttt{shorttitle} and \texttt{leftheader}, \texttt{authors\_note}, and
so forth, are implemented in APA style documents.

\begin{verbatim}
---
doctype: jou # this can be stu, jou, doc, or man
title: "Your title"
author: 
  - name: "Your name"
    affiliation_number: 1
    affiliation: "Your affiliation"
shorttitle: "APA short title"
leftheader: "Authors' last names (for jou doctype only)"
authors_note: |
  This is an author's note; 
  you could remove it if not desired. 
abstract: "Your article's abstract"
keywords: "example, keywords"
date: "3/29/21"
output: 
  pdf_document:
    template: "template.tex"
bibliography: example.bib # update to be your .bib file!
csl: apa.csl
---
\end{verbatim}

\begin{center}\rule{0.5\linewidth}{0.5pt}\end{center}

You needn't write anything below the title of your references
section---the bibliography will be automatically generated. If you
notice that something is inappropriately not capitalized, make sure it
is capitalized correctly in your .bib file; if it is, consider
surrounding it with curly brackets to ensure

\hypertarget{references}{%
\section*{References}\label{references}}
\addcontentsline{toc}{section}{References}

\hypertarget{refs}{}
\begin{CSLReferences}{1}{0}
\leavevmode\hypertarget{ref-austbarth2020}{}%
Aust, F., \& Barth, M. (2020). \emph{{papaja}: {Create} {APA}
manuscripts with {R Markdown}}. \url{https://github.com/crsh/papaja}

\end{CSLReferences}

\end{document}